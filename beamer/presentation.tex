% \documentclass[10pt,handout]{beamer}
\documentclass[10pt,xcolor=dvipsnames]{beamer}
\usepackage[german]{babel}
\usepackage{times}
\usepackage[T1]{fontenc}
\usetheme{Boadilla}
\setbeamercovered{transparent}
\usepackage{appendixnumberbeamer}
\setbeamertemplate{navigation symbols}{}
\setbeamertemplate{itemize items}[default]
\setbeamertemplate{enumerate items}[default]
\setbeamertemplate{itemize subitem}{\rule[0.5ex]{0.5ex}{0.6ex}}
\definecolor{UBCblue}{rgb}{0.04706, 0.13725, 0.26667} % UBC Blue (primary)
\usecolortheme[named=UBCblue]{structure}
\usefonttheme{serif}

\usepackage[utf8]{inputenc}
\usepackage[many]{tcolorbox}

% Pakete für Mathematikmodus:
\usepackage{amsmath, amssymb, nccmath}

% Zum Setzen von Einheiten
\usepackage{siunitx}
\sisetup{output-decimal-marker = {.},
            locale = US}
\DeclareSIUnit{\million}{\text{Mio.}}
\usepackage{physics}
\hypersetup{
    colorlinks=false,
    linkcolor=blue,
    filecolor=magenta,      
    urlcolor=cyan,
}
\usepackage{pdfpages}

\title[Thesis]{\vspace{2cm} \\ TITLE}
\subtitle{subtitle}
\author[M. Hartmann]{Maximilian Hartmann \\ %
	\texttt{\scriptsize example@mail.com}}
\institute[Group A]{institution}
\date[Februar 2021]{11. Februar 2021}

% Customisation: Table of Content
\setbeamertemplate{section in toc}{%
    \leavevmode\leftskip=1.75ex%
    \llap{%
        \usebeamerfont*{section number projected}%
        \usebeamercolor[bg]{section number projected}%
        \vrule width2.25ex height1.85ex depth.4ex%
        \hskip-2.25ex%
        \hbox to2.25ex{\hfill\color{fg}\inserttocsectionnumber\hfill}}%
        \kern1.25ex\inserttocsection\par}

% PGFplots and TikZ
\usepackage{pgfplots}
\DeclareUnicodeCharacter{2212}{−}
\usepgfplotslibrary{groupplots,dateplot}
\usetikzlibrary{arrows,decorations.markings,patterns}
\usetikzlibrary{decorations.pathmorphing,arrows,intersections}
\usetikzlibrary{pgfplots.fillbetween}
\usetikzlibrary{positioning}
\usetikzlibrary{spy}
\usetikzlibrary{calc}
\usetikzlibrary{through}
\pgfplotsset{compat=newest}
\pgfplotsset{every axis/.append style={
		%mark size=0.5pt,
		line width=0.5pt,
		tick style={line width=.6pt},
        	label style={font=\tiny},
		tick label style={font=\tiny},
		legend style={font=\tiny},
		}}
\pgfkeys{/pgf/number format/.cd,1000 sep={\,}}
% tikzpicture overlays
\tikzset{
    invisible/.style={opacity=0,text opacity=0},
    visible on/.style={alt={#1{}{invisible}}},
    alt/.code args={<#1>#2#3}{%
        \alt<#1>{\pgfkeysalso{#2}}{\pgfkeysalso{#3}} 
	% \pgfkeysalso doesn't change the path
        },
    }
\usepackage{pgffor}
\usepackage{adjustbox}
\usepackage{textpos}
\usetikzlibrary{external}
\tikzexternalize[prefix=tikzextern/]
\usepackage[mode=buildnew]{standalone}

%%%%%%%%%%%%%%%%%%%%%%%%%%%%%%%%%%%%%%%%%%%%%%%%%%%%%%%%%%%%%%%%%%%%%%%%%%%%%%%%%%%%%%%%%%
%%%%%%%%%%%%%%%%%%%%%%%%%%%%%%%%%%%%%%%%%%%%%%%%%%%%%%%%%%%%%%%%%%%%%%%%%%%%%%%%%%%%%%%%%%
%%%%%%%%%%%%%%%%%%%%%%%%%%%%%%%%%%%%%%%%%%%%%%%%%%%%%%%%%%%%%%%%%%%%%%%%%%%%%%%%%%%%%%%%%%
\begin{document}
\begin{frame}
\titlepage
\begin{textblock*}{2cm}(0.5cm,-8.5cm)
	optional logo could be planted here
\end{textblock*}
\end{frame}

\begin{frame}
\frametitle{Inhalt}
\tableofcontents
\end{frame}

%%%%%%%%%%%%%%%%%%%%%%%%%%%%%%%%%%%%%%%%%%%%%%%%%%%%%%%%%%%%%%%%%%%%%%%%%%%%%%%%%%%%%%%%%%
\section{Einleitung}
%%%%%%%%%%%%%%%%%%%%%%%%%%%%%%%%%%%%%%%%%%%%%%%%%%%%%%%%%%%%%%%%%%%%%%%%%%%%%%%%%%%%%%%%%%
% \subsection{Motivation}
% \subsection{Ziele}
\begin{frame}
    \frametitle{\secname}
\framesubtitle{Motivation und Ziele}
    \begin{itemize}
        \item Turbulenz hat maßgeblichen Einfluss auf Impuls- und Wärmetransport 
            \pause
        \item Allgemeiner Ansatz: statistische Beschreibung der Auswirkungen von Turbulenz \\
            \begin{itemize}
                \item subitem 1
                \item subitem 2
                    \pause
            \end{itemize}
    \end{itemize}
\end{frame}
%%%%%%%%%%%%%%%%%%%%%%%%%%%%%%%%%%%%%%%%%%%%%%%%%%%%%%%%%%%%%%%%%%%%%%%%%%%%%%%%%%%%%%%%%%
\section{Turbulenter Impulstransport}
%%%%%%%%%%%%%%%%%%%%%%%%%%%%%%%%%%%%%%%%%%%%%%%%%%%%%%%%%%%%%%%%%%%%%%%%%%%%%%%%%%%%%%%%%%
\subsection{SST-Modell}
\begin{frame}
    \frametitle{\secname}
    \framesubtitle{\subsecname}
    \begin{itemize}
        \item<1-> Modellierung des turbulenten Spannungstensors: $\tau_{t,ij}$
        \item<2-> Shear-Stress-Transport Modell (Menter, 2003)
        \item<3-> Kombination zwischen $k$-$\varepsilon$-Modell in wandferne und
            $k$-$\omega$-Modell in wandnähe (ohne CDT)
    \end{itemize}
    \onslide<4->{
    \small
    \begin{equation} \label{eq:kTransport}
        \frac{\text{D} \rho k}{\text{D} t} 
        =  
        P_k
        - \beta^* \rho \omega k 
        + \frac{\partial }{\partial x_j} \left[ \left( \mu + \sigma_{k} \mu_{t} \right) \frac{\partial k}{\partial x_j} \right]
    \end{equation}
    \begin{equation} \label{eq:omegaTransport}
        \begin{split}
            \frac{\text{D} \rho \omega}{\text{D} t} = 
            \frac{\gamma}{\nu_t} 
            P_k
            - \beta \rho \omega^2 + \frac{\partial }{\partial x_j} \left[ \left( \mu + \sigma_{\omega} \mu_{t} \right) \frac{\partial \omega}{\partial x_j} \right] \\
            + \underbrace{2 \rho (1-F_1) \sigma_{\omega2} \frac{1}{\omega} \frac{\partial k}{\partial x_j} \frac{\partial \omega}{\partial x_j}}_{CDT}
        \end{split}
    \end{equation}
    }
    \onslide<5->{
    \begin{columns}[c]
        \begin{column}{0.5\textwidth}
            \vspace{0.5cm}
            \begin{adjustbox}{max totalsize={1\textwidth}{.5\textheight},center} 
                % \documentclass{report}
% \usepackage{pgfplots}
% \usetikzlibrary{patterns}
% \usetikzlibrary{decorations.pathmorphing,arrows,intersections}
% \usetikzlibrary{pgfplots.fillbetween}
% \usetikzlibrary{positioning}
% 
% \begin{document}
\begin{tikzpicture}[scale=0.8, visible on=<4->]

% Plate
\filldraw[draw=black,thick,fill=cyan] (0,0) rectangle (10,3);

% layer 1 (F1=1)
\draw[name path=layer1] (2,0) .. controls (3,0.3) and (6,0.5) .. (10,0.6) ;
\draw[name path=bottomline1] (2,0) -- (10,0.0) node[below, pos=0.5] {No-Slip Wall};
\tikzfillbetween[of=layer1 and bottomline1]{red,opacity=0.5};

% layer 2 (Blending-Region)
\draw[name path=layer2] (1,0) .. controls (2,1.0) and (6,1.1) .. (10,1.3);
\draw[name path=bottomline2,line width=0.7mm,red] (2,0) -- (10,0.0);
\tikzfillbetween[of=layer2 and bottomline2]{red, opacity=0.5};
\draw[line width=0.7mm,black] (0,0) -- (2,0);

% Beschriftungen
\node[align=left,anchor=north] at (1,0) {Slip Wall};
\node[align=left,anchor=north] at (6,2) {$k-\varepsilon$ ($F_1=0$)};
\draw[name path=layer1] (2,0) .. controls (3,0.3) and (6,0.5) .. (10,0.6) 
    node[below,pos=0.8,sloped] {$k-\omega$ ($F_1=1$)};
\draw[black] (1,0) .. controls (2,1.0) and (6,1.1) .. (10,1.3) 
    node[below, pos=0.7, sloped] {Blending Region};

    \draw[pattern=north east lines, pattern color=black] 
        (2,0) rectangle (10,-0.15);

% inlet profile
\filldraw[draw=black,thick,fill=cyan] (-1,0) rectangle (0,3);
\foreach \y in {0,0.3,...,3.01}{
    \draw[-stealth] (-1,\y) -- +(1,0);}
\end{tikzpicture}
% \end{document}

            \end{adjustbox}
        \end{column}
        \begin{column}{0.50\textwidth}
            \centering
            \begin{equation}\label{eq:F1}
                \phi = F_1 \phi_1 + (1 - F_1) \phi_2
            \end{equation}
            \scriptsize
            \begin{tabular}{|c|c|c|c|c|c|c|}
                \hline 
                $\phi$ & $\beta$ & $\sigma_k$ & $\sigma_\omega$ & $\gamma$ & $a_1$ & $\beta^*$ \\ 
                \hline 
                $\phi_1$ & $0.075$ & $0.5$ & $0.5$ & $5/9$ & $0.31$ & $0.09$ \\ 
                \hline 
                $\phi_2$ & $0.0828$ & $1$ & $0.856$ & $0.44$ & $0.31$ & $0.09$ \\ 
                \hline 
            \end{tabular} 
            \normalsize
        \end{column}
    \end{columns}
}
\end{frame}

%%%%%%%%%%%%%%%%%%%%%%%%%%%%%%%%%%%%%%%%%%%%%%%%%%%%%%%%%%%%%%%%%%%%%%%%%%%%%%%%%%%%%%%%%%
\appendix
\renewcommand{\theframenumber}{\alph{framenumber}}

% Literaturverzeichnis:
% \begin{frame}[allowframebreaks]{References}
%     \def\newblock{}
%     \bibliographystyle{plain}
%     \bibliography{mybib}
% \end{frame}

% \section{Randbedingungen}


%%%%%%%%%%%%%%%%%%%%%%%%%%%%%%%%%%%%%%%%%%%%%%%%%%%%%%%%%%%%%%%%%%%%%%%%%%%%%%%%%
% Modellkonstanten: ktOmt
%%%%%%%%%%%%%%%%%%%%%%%%%%%%%%%%%%%%%%%%%%%%%%%%%%%%%%%%%%%%%%%%%%%%%%%%%%%%%%%%%
\begin{frame}
    \frametitle{Anhang}
    \framesubtitle{Überblick der Modellkonstanten}
    Zusammenfassen der Konstanten: \\
    \begin{itemize}
        \item $C_{P1} = \left( \frac{C_{P1} f_{P1}}{2}  - 1 \right)_{AKN}$
        \item $C_{D1} = \left( \frac{C_{D1} f_{D1}}{2}  - 1 \right)_{AKN}$
        \item $C_{P2} = \left( C_{P2} f_{P2} \right)_{AKN}$
        \item $C_{D2} = \left( C_{D2} f_{D2} \right)_{AKN}$
    \end{itemize}
    \vspace{1cm}
    \begin{scriptsize}
        \begin{table}
            \centering
            \begin{tabular}{|c|c|c|c|c|c|c|c|c|c|}
                \hline
                Autor & $C_\lambda$ & $C_{P1}$ & $C_{D1}\beta_\theta^*$ & $C_{P2}$ & $C_{D2}\beta^*$ & $\sigma_{k_\theta}$ & $\sigma_{\omega_\theta}$  & $C_\mu=\beta^*$ & $\beta_\theta^*$ \\ 
                \hline  \hline 
                Rochhausen & $0.147$ & $-0.35$ & $0.07$ & $0.28$ & $0.05$ & $5$ & $5$ & $0.09$ & $0.135$ \\ 
                \hline 
                Cerroni & $0.1$ & $0.1$ & $0.036$ & $0.6$ & $0.072$ & $1.4$ & $1.4$ & $0.09$ & $0.09$ \\ 
                \hline 
                Abe & $0.1$ & $-0.05$ & $0$ & $0.6$ & $0.081$ & $1.6$ & $1.6$ & $0.09$ & $0.09$ \\ 
                \hline 
            \end{tabular}
            \caption{Modellkonstanten des Wirbelleitfähigkeitsmodells}
        \end{table}
    \end{scriptsize}
\end{frame}
%%%%%%%%%%%%%%%%%%%%%%%%%%%%%%%%%%%%%%%%%%%%%%%%%%%%%%%%%%%%%%%%%%%%%%%%%%%%%%%%%
% Wandrandbedingungen
%%%%%%%%%%%%%%%%%%%%%%%%%%%%%%%%%%%%%%%%%%%%%%%%%%%%%%%%%%%%%%%%%%%%%%%%%%%%%%%%%
\begin{frame}
    \frametitle{Anhang}
    \framesubtitle{Wandrandbedingung}
    \subsection{Wand}
    Wandrandbedingung aus Abe:
    \begin{equation}
        \varepsilon_{\theta,wall} = 
        \alpha \left( \frac{\partial \sqrt{\overline{T'^2}}}{ \partial n} \right)^2
        \approx \frac{\nu_w}{Pr} \frac{\overline{T'^2}}{y^2}
    \end{equation}
    mit $\alpha = \nu/\Pr$, $T'^2 = 2k_{\theta}$ und $\varepsilon_{\theta,wall}=\beta_\theta^* k_{\theta} \omega_{\theta,wall} $:
    \begin{equation}
        \omega_{\theta,wall} 
        = \frac{\nu_w}{Pr} \frac{2 k_{\theta}}{\beta_\theta^* k_{\theta} y^2}
        = \frac{\nu_w}{Pr} \frac{2 }{\beta_\theta^* y^2}
    \end{equation}
\end{frame}
\end{document}%
